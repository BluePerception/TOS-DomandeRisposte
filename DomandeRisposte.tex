\documentclass[a4paper]{article}
\usepackage[T1]{fontenc}
\usepackage[utf8]{inputenc}
\usepackage[italian]{babel}


\begin{document}

	\section{Introduzione al copyright}
	
		\paragraph{Raccontare gli eventi che portarono la nascita degli editti a Venezia (proto-copyright)}
		
		\paragraph{Chi è Johames di Spira?} % Domanda incomprensibile
		
		\paragraph{Qual è uno dei grandi cambiamenti dell'editto di Ann?}
	
	
	\section{Nascita del progetto GNU}

		\paragraph{Chi è Richard Stallman?}
	
		\paragraph{Dove aveva cominciato a lavorare prima del MIT?}
	
		\paragraph{Com'è che Stallman entra nella comunità del MIT?}
	
		\paragraph{Qual è il lavoro di Stallman?}
	
		\paragraph{Su cosa comincerà a lavorare Stallman nel MIT?}
	
	
		\paragraph{Che cos'è TECO?}
	
		\paragraph{Sotto quale licenza era TECO?}
	
		\paragraph{Che cos'è Emacs?} %GNU HURD ???
	
		\paragraph{Che cosa permetteva di fare Emacs?}
	
		\paragraph{Sotto quale licenza venne distribuito Emacs?}

		\paragraph{Che relazione c'è tra TECO e Emacs?}
		
		
		\paragraph{Durante la fine del periodo hacker (anni '70), che cosa fa crollare tutto? Chi e perché si organizza uno sciopero?}
		
		
		\paragraph{Chi ha inventato le Lisp machine? Cosa sono? Quali aziende le hanno inventate? Perché queste aziende hanno avuto un impatto sulla comunità hacker?}
	
	

	\section{Linux}
		
		\paragraph{Racconta la nascita di Linux partendo da il sistema UNIX passando per MINIX}
		
		\paragraph{Come si è evoluto e diffuso Linux?}
	
	
	\section{Open source}	
		\paragraph{Come è nato il movimento Open-Source?}: \\
		Il movimento Open Source nasce ovviamente dai principi già affermati ed esistenti del \textit{free software}. Negli anni '90, il movimento del free software non era ben visto nel mondo commerciale/business in quanto le grandi aziende mal vedevano questo filone culturale. Probabilmente anche il termine \textit{free} ha contribuito a ciò, in quanto spesso viene confuso con "gratuito" quando, invece, sta per "libero". \\
		Nel 1997, Eric Raymond(informatico statunitense) scrisse "\textit{The Cathedral and the Bazaar}", cioè un saggio che analizza Linux e la sua architettura, sostenendo che quest'ultimo fosse destinato al collasso e al fallimento. Raymond sosteneva che, nella comunità degli sviluppatori che si dedicavano a Linux, ognuno avrebbe (potuto fare) fatto di testa sua, causando lo "scoppio" di Linux. Il punto è che ciò non accadeva e Raymond, nel suo saggio, ne analizza il motivo.\\
		Il risultato dell'analisi fu una serie di linee guida che, in qualche modo, gli sviluppatori tendevano a seguire.
		Raymond elencò e spiegò quali sono quindi le regole che una comunità di sviluppatori dovrebbero seguire per non far "esplodere/collassare" il loro progetto:
		\begin{itemize}
			\item ogni progetto dev'essere un "prurito" del programmatore;
			\item i buoni programmatori sanno cosa scrivere, i grandi programmatori sanno cosa RIscrivere;
			\item tratta i tuoi utenti come co-sviluppatori, come se il programma fosse stato fatto insieme a loro. Questo porta molte persone a migliorare il software, in quanto esse si sentono parte del progetto.  Inoltre, la varietà di persone porta ad una varietà di soluzioni allo stesso problema con approcci diversi, e la combinazione dei contributi può significare grossi miglioramenti.
			\item distribuisci presto, spesso e presta ascolto ai tuoi utenti, in modo da trovare e risolvere i bachi velocemente.
			\item «Dato un numero sufficiente di occhi, tutti i bug vengono a galla», Eric Raymond. Quindi dato un sufficiente numero di beta tester ogni 
			problema verrà identificato e risolto.
			\item Riconoscere le buone idee degli utenti è importante 
			come averne di proprie.
		\end{itemize}
		Il saggio di Raymond riscosse molto successo(rendendo l'autore famoso) non solo tra gli sviluppatori, ma anche tra alcune grandi aziende, uno tra esse Netscape.
		Ispirata dal saggio di Raymond, nel 1998 Netscape rilasciò il codice sorgente del proprio browser.
		In realtà, Netscape decise di pubblicare il codice anche per un altro motivo: in quel periodo, Microsoft stava sviluppando un suo browser e a Netscape pensavano che sarebbe potuto diventare dominante nel mercato, destinando Netscape alla deriva. Proprio per questo, rilasciarono il sorgente sotto licenza libera. Su questa decisione, ebbe un'importante influenza Raymond. \\
		Per i motivi elencati inizialmente, c'era però il rischio che il termine "free software" non assumesse la giusta importanza(e di conseguenza Netscape che aveva dichiarato il suo browser free software). Venne coniato così, all'interno di Netscape, il termine "\textit{Open Source}". Di conseguenza, nel 1998 Eric Raymond e Bruce Perens fondarono l'OSI(Open Source Initiative), avente lo scopo di conciliare la filosofia del free software con le regole di mercato.
		
		
		\paragraph{Chi scrive "La cattedrale e il bazaar"? Di cosa parla? In quali anni?}: \\
		Vedi domanda precedente. \\
		\textbf{--- breve spiegazione del titolo del saggio ---} \\
		Nel modello a cattedrale il programma viene realizzato da un numero limitato di "esperti" che provvedono a scrivere il codice in quasi totale isolamento. Il progetto ha una suddivisione gerarchica molto stretta e ogni sviluppatore si preoccupa della sua piccola parte di codice. Le revisioni si susseguono con relativa lentezza e gli sviluppatori cercano di distribuire programmi il più possibile completi e senza bug. Il programma Emacs, il GCC e molti altri programmi si basano su questo modello di sviluppo.\\
		Nel modello a bazaar il codice sorgente della revisione in sviluppo è disponibile liberamente, gli utenti possono interagire con gli sviluppatori e se ne hanno le capacità possono modificare e integrare il codice. Lo sviluppo è decentralizzato e non esiste una rigida suddivisione dei compiti, un programmatore di buona volontà può modificare e integrare qualsiasi parte del codice. In sostanza lo sviluppo è molto più diluito e libero, da qui il nome di modello a bazaar. Il kernel Linux e molti programmi utilizzano questo nuovo modello di sviluppo associativo.
		\\
		\textbf{--- fine ---}
		
		\paragraph{Quale grande evento ha fatto si che Raymond diventasse famoso?\\}
		Sia per il saggio pubblicato che per la decisione di rilasciare il sorgente di Netscape(principalmente per quest'ultimo).
		
		\paragraph{A quali scopi Netscape rilascia il codice sorgente?}
		: \\ vedi domanda precedente
		\paragraph{Cosa relaziona il rilascio del codice sorgente di Netscape a Raymond?}
		 vedi domanda precedente
		
	\section{Creative Commons}
	
		\paragraph{Come nasce il primo movimento di Creative Commons?}
		
		\paragraph{Qualcuno ha guidato il movimento Open Content? Chi sono state le prime persone che si sono battute per questi diritti?}
		
		
		
		
	\section{Licenze}
	
		\subsection{GPL v2}
		
			\paragraph{Illustrare tutti gli obblighi che una persona deve soddisfare per poter distribuire solo i binari 	(nel testo)}
		
			\paragraph{Nel caso di sorgenti modificati quali sono gli obblighi? (Illustrare nel testo)}
		
			\paragraph{Cosa fare nel caso dovessi passare la libreria a una terza persona? Ci sono limitazioni?}
			
			\paragraph{Come rendere disponibile la consultazione dei codici sorgenti di binari con licenza GPL v2?} % 2 possibilità
			
			\paragraph{Quali sono le 3 condizioni che si applicano per chi distribuisce solo i binari?}
			
			\paragraph{Cosa contiene la clausola "Libertà o morte"?}
		
		\subsection{GPL v3}
			
			\paragraph{Illustrare le caratteristiche principali}
		
		\subsection{BSD}
		
			\paragraph{Illustrare brevemente 2a, 3a e 4a clausola}
			
		\subsection{Apache}
		
			\paragraph{Spiegare la clausola di rinomina}
			
	
		\subsection{Mozilla public license}
		
			\paragraph{Spiegare la licenza}
		
		
	\section{Date importanti}
	
	\begin{description}
		\item[1700:] anni degli editti a Venezia (proto-copyright)
		\item[1710:] editto di Ann
		\item[1976:] Copyright UA
		\item[1983:] Manifesto GNU di Stallman
		\item[1991:] Prima versione di Linux
	\end{description}
	
	% --- STRUMENTI OPEN SOURCE --- %
	
	\section{SVN}
	
	\paragraph{Creare un repository SVN}
	
	\paragraph{Creare una working copy on checkout}
	
	\paragraph{Mostrare un ciclo modifica -> commit -> update}
	
	\paragraph{Mostrare l'uso dei comandi svn status e revert <file>}
	
	\paragraph{Quali sono le opzioni di svn status?}
	
	\paragraph{Eseguire il backup del repository}
	
	\paragraph{Cancellare un repository}
	
	\paragraph{Ripristinare il repository}
	
	\paragraph{Mostrare come riconoscere se un file è stato modificato}
	
	\paragraph{Eseguire un commit ignorando alcuni file}
	
	\paragraph{Ignorare tutti i file all'interno di una precisa cartella}
	
	\paragraph{Ottenere conflitti tra due utenti}
	
	\paragraph{Risolvere manualmente un conflitto e committare}
	
	\paragraph{Cancellare le modifiche dopo averle committate}
	
	\paragraph{Caricare il repository in Internet}
	
	\paragraph{Accedere al repository da remoto}
	
	\paragraph{Creare dei permessi per l'accesso al repository}
	
	\paragraph{Copiare il contenuto di una cartella tranne del repository in una cartella branches/stable. Inserire un messaggio}

	\paragraph{Illustrare le conseguenze dell'uso del comando svn update -r}
	
	\paragraph{Mostrare come risolvere l'errore "out-of-date"}	
	

	
	\section{Mercurial} % Lo chiede ???
	
		\paragraph{Creare un repository Mercurial}
		
		\paragraph{Operazioni per configurare un repository Mercurial}
		
		\paragraph{Aggiungere un file al repository e poi commentarlo}	
		
		\paragraph{Eseguire un commit}
		
		\paragraph{Fare una modifica ad un file di testo e committarlo}
		
		\paragraph{Fare revert di una modifica committata}
		
		\paragraph{Rimuovere una cartella e committare}
		
		
	
\end{document}