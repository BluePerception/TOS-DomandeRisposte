\documentclass[a4paper]{article}
\usepackage[T1]{fontenc}
\usepackage[utf8]{inputenc}
\usepackage[italian]{babel}


\begin{document}

	\section{Introduzione al copyright}
	
		\paragraph{Raccontare gli eventi che portarono la nascita degli editti a Venezia (proto-copyright)}
		
		\paragraph{Chi è Johames di Spira?} % Domanda incomprensibile
		
		\paragraph{Qual è uno dei grandi cambiamenti dell'editto di Ann?}
	
	
	\section{Nascita del progetto GNU}

		\paragraph{Chi è Richard Stallman?}: \\
		Nato nel 1953, Stallman è uno dei principali esponenti del movimento del software libero.
		Megli anni '70, Stallman entra a far parte dell'IBM New York Scientific Center, un gruppo di ricerca nell'ambito della computer science. Stallman passò qui l'estate dopo il diploma di scuola superiore sviluppando programmi in Fortran per il calcolo numerico. Al tempo non era ancora un informatico, ovviamente, ma se ne voleva interessare.\\
		Nel 1971 entra ad Harvard, ma si interessa di più al MIT di Boston. Infatti Nel giugno del 1971 al primo anno da studente all'università Harvard, Stallman diventò un programmatore al laboratorio IA (Intelligenza Artificiale) del MIT, assunto da Russ Noftsker come programmatore di sistema. Da qui, prese parte alla comunità degli hacker, lavorando insieme a Richard Greenblatt e Bill Gosper. \\
		Nel 1980, al MIT c'era una stampante(Xerox 9700) al servizio degli utenti. Essa però non aveva un buon driver e alcuni messaggi d'errore non erano comunicati. In particolare, quando essa si inceppava, l'unico modo per scoprirlo era andare a verificare di persona, perdendo del tempo inutilmente. Essendo programmatore di sistema, Stallman voleva fixare questo problema, ma per farlo doveva avere a disposizione il codice sorgente della stampante in questione. Stallman fece richiesta, ma gli venne negato l'accesso al codice sorgente, impedendogli di fixare il bug.\\
		Questo fu un fatto che spinse successivamente Stallman a fondare il progetto GNU(GNU is Not Unix).
		
		\paragraph{Dove aveva cominciato a lavorare prima del MIT?}
		vedi domanda sopra
		\paragraph{Com'è che Stallman entra nella comunità del MIT?}
		vedi domanda sopra
		\paragraph{Qual è il lavoro di Stallman?}
		credo ci si riferisca quando è al MIT, vedi domanda sopra(programmatore di sistema)
		\paragraph{Su cosa comincerà a lavorare Stallman nel MIT?}
		vedi domanda sopra
		\paragraph{Che cos'è TECO?}: \\
		Text Editor and COrrector, è in qualche modo l'antenato di Emacs.\\
		Esso è uno dei primissimo programmi per editing di testo usabili tramite un monitor(e non più tramite telescrivente). Ad essere precisi, TECO è simile a un linguaggio di programmazione interpretato, mirato alla manipolazione del testo. Praticamente ogni carattere è un comando (una sequenza di uno o due caratteri rimpiazza le usuali parole chiave di linguaggi più verbosi) e quindi ogni stringa di caratteri è un programma TECO, anche se non necessariamente un programma utile. È sempre stato considerato molto complicato e, proprio per questo motivo, Stallman decise di cercare un qualche altro editor. \\
		Non trovando risultati soddisfacenti, Stallman scrisse una macro per TECO che permetteva di:
		\begin{itemize}
			\item editare il testo real time(prima non era possibile);
			\item random access editing(non ho ben capito cosa sia, credo sia l'opportunità di modificare qualsiasi punto del testo in tempo costante \textit{O}(1));
			\item aggiungere altre macro.
		\end{itemize}
		Cominciarono a nascere diverse macro per TECO, col fine di migliorarlo.
		Ad un certo punto, però, cominciarono a diventare troppe e disordinate. Guy Steele ebbe in seguito l'idea di fare un po' di ordine, cosa che poi Stallman prese a cuore e continuò. Le macro vennero tra loro omogeneizzate e un disordine del genere non sarebbe più dovuto succedere in futuro. \textbf{da qui parte la storia di Emacs...}
	
		\paragraph{Sotto quale licenza era TECO?}:\\
		non ho trovato nulla per ora
	
		\paragraph{Che cos'è Emacs?} %GNU HURD ???
	
		\paragraph{Che cosa permetteva di fare Emacs?}
	
		\paragraph{Sotto quale licenza venne distribuito Emacs?}

		\paragraph{Che relazione c'è tra TECO e Emacs?}
		
		
		\paragraph{Durante la fine del periodo hacker (anni '70), che cosa fa crollare tutto? Chi e perché si organizza uno sciopero?}
		
		
		\paragraph{Chi ha inventato le Lisp machine? Cosa sono? Quali aziende le hanno inventate? Perché queste aziende hanno avuto un impatto sulla comunità hacker?}
	
	

	\section{Linux}
		
		\paragraph{Racconta la nascita di Linux partendo da il sistema UNIX passando per MINIX}
		
		\paragraph{Come si è evoluto e diffuso Linux?}
	
	
	\section{Open source}
	
		\paragraph{Come è nato il movimento Open-Source?}
		
		\paragraph{Chi scrive "La cattedrale e il bazaar"? Di cosa parla? In quali anni?}
		
		\paragraph{Quale grande evento ha fatto si che Raymond diventasse famoso?}
		
		\paragraph{A quali scopi Netscape rilascia il codice sorgente?}
		
		\paragraph{Cosa relaziona il rilascio del codice sorgente di Netscape a Raymond?}
		
		
	\section{Creative Commons}
	
		\paragraph{Come nasce il primo movimento di Creative Commons?}
		
		\paragraph{Qualcuno ha guidato il movimento Open Content? Chi sono state le prime persone che si sono battute per questi diritti?}
		
		
		
		
	\section{Licenze}
	
		\subsection{GPL v2}
		
			\paragraph{Illustrare tutti gli obblighi che una persona deve soddisfare per poter distribuire solo i binari 	(nel testo)}
		
			\paragraph{Nel caso di sorgenti modificati quali sono gli obblighi? (Illustrare nel testo)}
		
			\paragraph{Cosa fare nel caso dovessi passare la libreria a una terza persona? Ci sono limitazioni?}
			
			\paragraph{Come rendere disponibile la consultazione dei codici sorgenti di binari con licenza GPL v2?} % 2 possibilità
			
			\paragraph{Quali sono le 3 condizioni che si applicano per chi distribuisce solo i binari?}
			
			\paragraph{Cosa contiene la clausola "Libertà o morte"?}
		
		\subsection{GPL v3}
			
			\paragraph{Illustrare le caratteristiche principali}
		
		\subsection{BSD}
		
			\paragraph{Illustrare brevemente 2a, 3a e 4a clausola}
			
		\subsection{Apache}
		
			\paragraph{Spiegare la clausola di rinomina}
			
	
		\subsection{Mozilla public license}
		
			\paragraph{Spiegare la licenza}
		
		
	\section{Date importanti}
	
	\begin{description}
		\item[1700:] anni degli editti a Venezia (proto-copyright)
		\item[1710:] editto di Ann
		\item[1976:] Copyright UA
		\item[1983:] Manifesto GNU di Stallman
		\item[1991:] Prima versione di Linux
	\end{description}
	
	% --- STRUMENTI OPEN SOURCE --- %
	
	\section{SVN}
	
	\paragraph{Creare un repository SVN}
	
	\paragraph{Creare una working copy on checkout}
	
	\paragraph{Mostrare un ciclo modifica -> commit -> update}
	
	\paragraph{Mostrare l'uso dei comandi svn status e revert <file>}
	
	\paragraph{Quali sono le opzioni di svn status?}
	
	\paragraph{Eseguire il backup del repository}
	
	\paragraph{Cancellare un repository}
	
	\paragraph{Ripristinare il repository}
	
	\paragraph{Mostrare come riconoscere se un file è stato modificato}
	
	\paragraph{Eseguire un commit ignorando alcuni file}
	
	\paragraph{Ignorare tutti i file all'interno di una precisa cartella}
	
	\paragraph{Ottenere conflitti tra due utenti}
	
	\paragraph{Risolvere manualmente un conflitto e committare}
	
	\paragraph{Cancellare le modifiche dopo averle committate}
	
	\paragraph{Caricare il repository in Internet}
	
	\paragraph{Accedere al repository da remoto}
	
	\paragraph{Creare dei permessi per l'accesso al repository}
	
	\paragraph{Copiare il contenuto di una cartella tranne del repository in una cartella branches/stable. Inserire un messaggio}

	\paragraph{Illustrare le conseguenze dell'uso del comando svn update -r}
	
	\paragraph{Mostrare come risolvere l'errore "out-of-date"}	
	

	
	\section{Mercurial} % Lo chiede ???
	
		\paragraph{Creare un repository Mercurial}
		
		\paragraph{Operazioni per configurare un repository Mercurial}
		
		\paragraph{Aggiungere un file al repository e poi commentarlo}	
		
		\paragraph{Eseguire un commit}
		
		\paragraph{Fare una modifica ad un file di testo e committarlo}
		
		\paragraph{Fare revert di una modifica committata}
		
		\paragraph{Rimuovere una cartella e committare}
		
		
	
\end{document}