\documentclass[a4paper]{article}
\usepackage[T1]{fontenc}
\usepackage[utf8]{inputenc}
\usepackage[italian]{babel}


\begin{document}
	\section{Credits}
	Domande fornite da Eduard Bicego. \\
	Risposte fornite da Mattia Bottaro, Luca Bertolini.
	
	\section{Introduzione al copyright}
	
	\paragraph{Raccontare gli eventi che portarono la nascita degli editti a Venezia (proto-copyright)}
	
	\paragraph{Chi è Johames di Spira?} % Domanda incomprensibile
	
	\paragraph{Qual è uno dei grandi cambiamenti dell'editto di Ann?}
	
	
	\section{Nascita del progetto GNU}
	
	\paragraph{Chi è Richard Stallman?}
	È uno dei principale esponenti del software libero bla bla...
	\paragraph{Dove aveva cominciato a lavorare prima del MIT?}
	
	\paragraph{Com'è che Stallman entra nella comunità del MIT?}
	
	\paragraph{Qual è il lavoro di Stallman?}
	
	\paragraph{Su cosa comincerà a lavorare Stallman nel MIT?}
	
	
	\paragraph{Che cos'è TECO?}
	
	\paragraph{Sotto quale licenza era TECO?}
	
	\paragraph{Che cos'è Emacs?} %GNU HURD ???
	
	\paragraph{Che cosa permetteva di fare Emacs?}
	
	\paragraph{Sotto quale licenza venne distribuito Emacs?}
	
	\paragraph{Che relazione c'è tra TECO e Emacs?}
	
	
	\paragraph{Durante la fine del periodo hacker (anni '70), che cosa fa crollare tutto? Chi e perché si organizza uno sciopero?}
	
	
	\paragraph{Chi ha inventato le Lisp machine? Cosa sono? Quali aziende le hanno inventate? Perché queste aziende hanno avuto un impatto sulla comunità hacker?}
	
	
	
	\section{Linux}
	
	\paragraph{Racconta la nascita di Linux partendo da il sistema UNIX passando per MINIX}
	
	\paragraph{Come si è evoluto e diffuso Linux?}
	
	
	\section{Open source}
	
	\paragraph{Come è nato il movimento Open-Source?}
	
	\paragraph{Chi scrive "La cattedrale e il bazaar"? Di cosa parla? In quali anni?}
	
	\paragraph{Quale grande evento ha fatto si che Raymond diventasse famoso?}
	
	\paragraph{A quali scopi Netscape rilascia il codice sorgente?}
	
	\paragraph{Cosa relaziona il rilascio del codice sorgente di Netscape a Raymond?}
	
	
	\section{Creative Commons}
	
	\paragraph{Come nasce il primo movimento di Creative Commons?}
	
	\paragraph{Qualcuno ha guidato il movimento Open Content? Chi sono state le prime persone che si sono battute per questi diritti?}
	
	
	
	
	\section{Licenze}
	
	\subsection{GPL v2}
	
	\paragraph{Illustrare tutti gli obblighi che una persona deve soddisfare per poter distribuire solo i binari 	(nel testo)}
	
	\paragraph{Nel caso di sorgenti modificati quali sono gli obblighi? (Illustrare nel testo)}
	
	\paragraph{Cosa fare nel caso dovessi passare la libreria a una terza persona? Ci sono limitazioni?}
	
	\paragraph{Come rendere disponibile la consultazione dei codici sorgenti di binari con licenza GPL v2?} % 2 possibilità
	
	\paragraph{Quali sono le 3 condizioni che si applicano per chi distribuisce solo i binari?}
	
	\paragraph{Cosa contiene la clausola "Libertà o morte"?}
	
	\subsection{GPL v3}
	
	\paragraph{Illustrare le caratteristiche principali}: \\
	\subsection{BSD}
	
	\paragraph{Illustrare brevemente 2a, 3a e 4a clausola}

	\section{Date importanti}
	
	\begin{description}
		\item[1700:] anni degli editti a Venezia (proto-copyright)
		\item[1710:] editto di Ann
		\item[1976:] Copyright UA
		\item[1983:] Manifesto GNU di Stallman
		\item[1991:] Prima versione di Linux
	\end{description}
	
	% --- STRUMENTI OPEN SOURCE --- %
	
	\section{SVN}
	
	\paragraph{Creare un repository SVN}: \\
	Subversion è un sistema di versionamento centralizzato, ergo è necessario che il repository alloggi su un server. Noi lavoreremo in locale, ergo il server è \textit{localhost}.
	Nel server, si dovrà dare il comando:
	\begin{verbatim}
	svnadmin create <path>
	\end{verbatim}
	
	\paragraph{Creare una working copy on checkout}:\\
	Il comando \textit{checkout} serve per crearsi in locale una copia del progetto. È lo stesso concetto ci \textit{clone} di git.
	\begin{verbatim}
	svn checkout <url> <path>
	\end{verbatim}
	dove \textit{<url>} è la posizione del repository, mentre \textit{<path>} è il percorso LOCALE nel quale vogliamo copiarlo.
	Poniamo ad esempio di avere un repository nella cartella \textit{folder}, situata nel localhost.
	In locale, daremo il seguente comando:
	\begin{verbatim}
	svn checkout localhost/folder <path>
	\end{verbatim}
	Rispetto a clone di git, con checkout di svn è possibile copiare delle cartelle o singoli file del repo, semplicemente specificandone il percorso nel server.
	
	\paragraph{Mostrare un ciclo modifica -> commit -> update}
	\begin{itemize}
		\item modifica: consiste nella modifica di un file;
		\item commit: questo comando invia le modifiche della working copy al server.\begin{verbatim}
		svn commit --message "Messaggio del commit" [<path>]
		\end{verbatim}
		Il messaggio è obbligatorio e path è il percorso LOCALE dal quale caricare le modifiche. Se path non è specificato, allora assume il valore di default, ovvero la cartella corrente.
		\item update: questo comando aggiorna la working copy con la copia aggiornata del repository.
		\begin{verbatim}
		svn update
		\end{verbatim}
	\end{itemize}
	
	\paragraph{Mostrare l'uso dei comandi svn status e revert <file>}
	\begin{itemize}
		\item status: comando per controllare modifiche al repo e visualizzare informazioni su quest'ultimo.
		\begin{verbatim}
		svn status
		\end{verbatim}
		\item revert: vengono annullate le ultime modifiche in locale, in particolare i file in locale tornano alla stessa versione del server.
		\begin{verbatim}
		svn revert
		\end{verbatim}
	\end{itemize}
	\paragraph{Quali sono le opzioni di svn status?}
	\begin{itemize}
		
		
		\item - -changelist (- -cl) ARG
		\item - -depth ARG
		\item - -ignore-externals
		\item - -incremental
		\item - -no-ignore
		\item - -quiet (-q)
		\item - -show-updates (-u)
		\item - -verbose (-v)
		\item - -xml
		
	\end{itemize}
	da definire...
	\paragraph{Eseguire il backup del repository}: \\
	Si può fare in diversi modi:
	\begin{itemize}
		\item hotcopy con svnadmin: viene creata una copia di backup comprendente i file di configurazione, quelli del DB e gli hooks(programmi attivati da certi eventi sul repository). \begin{verbatim}
		svnadmin hotcopy <path_repo> <new_path_repo>
		\end{verbatim}
		\item dump con svnadmin: scrive(dump) il contenuto del filesystem in un "dump file" portable format tramite uno stream di output stdout.\begin{verbatim}
		svnadmin dump <path> > full.dump
		\end{verbatim}
		
		\item migrare un progetto ad un nuovo repo...?
		
	\end{itemize}
	\paragraph{Cancellare un repository}: \\
	Lo si può fare con i comandi non specifici di svn.
	\begin{verbatim}
	rm -rf <path>
	\end{verbatim}
	
	\paragraph{Ripristinare il repository}
	\begin{itemize}
		\item load con svnadmin: legge da uno stream di input stdin il dump di un repository
		\begin{verbatim}
		svnadmin load <path> < repos_backup
		\end{verbatim}
	\end{itemize}
	\paragraph{Mostrare come riconoscere se un file è stato modificato}
	
	\paragraph{Eseguire un commit ignorando alcuni file}: \\
	Si possono commitare file singoli. Poniamo di voler committare tutti i file .html e non quelli .php. Il comando sarà:
	\begin{verbatim}
	svn commit -m "messaggio di commit" *.html
	\end{verbatim}
	\paragraph{Ignorare tutti i file all'interno di una precisa cartella}
	
	\paragraph{Ottenere conflitti tra due utenti}
	
	\paragraph{Risolvere manualmente un conflitto e committare}
	
	\paragraph{Cancellare le modifiche dopo averle committate}
	
	\paragraph{Caricare il repository in Internet}
	
	\paragraph{Accedere al repository da remoto}
	
	\paragraph{Creare dei permessi per l'accesso al repository}
	
	\paragraph{Copiare il contenuto di una cartella tranne del repository in una cartella branches/stable. Inserire un messaggio}
	
	\paragraph{Illustrare le conseguenze dell'uso del comando svn update -r}
	
	\paragraph{Mostrare come risolvere l'errore "out-of-date"}	
	
	
	
	\section{Mercurial} % Lo chiede ???
	
	\paragraph{Creare un repository Mercurial}
	
	\paragraph{Operazioni per configurare un repository Mercurial}
	
	\paragraph{Aggiungere un file al repository e poi commentarlo}	
	
	\paragraph{Eseguire un commit}
	
	\paragraph{Fare una modifica ad un file di testo e committarlo}
	
	\paragraph{Fare revert di una modifica committata}
	
	\paragraph{Rimuovere una cartella e committare}
	
	
	
\end{document}