\documentclass[a4paper]{article}
\usepackage[T1]{fontenc}
\usepackage[utf8]{inputenc}
\usepackage[italian]{babel}


\begin{document}
	\section{Credits}
	Domande fornite da Emanuele Carraro. \\
	Domande trascritte da Eduard Bicego. \\
	Risposte fornite da Eduard Bicego, Luca Bertolini, Mattia Bottaro, Marco Zanella.
	
	\section{Introduzione al copyright}
	
		\paragraph{Raccontare gli eventi che portarono la nascita degli editti a Venezia (proto-copyright)}:\\
		Un editto è un'ordinanza emanata da un'autorità.
		Il copyright è l'insieme delle leggi sul diritto d'autore.
		Per proto-copyright s'intende l'insieme di privilegi ed editti che furono stabiliti ed emanati nella Venezia a cavallo del 1400 e del 1500, al fine di proteggere le opere, le modifiche ad esse, i loro autori e i meccanismi di produzione.
		In realtà, il primo privilegio concesso a Venezia è stato a Johames di Spira(1469), il quale introdusse la stampa a Venezia. Il privilegio, della durata di 5 anni, impediva a chiunque(se non a J. di Spira) di esercitare la stampa o di importare libri all'estero. Esso, però, morì poco dopo che il privilegio gli venne concesso, annullandolo. \\
		Si noti che questo privilegio protegge il meccanismo di produzione, non l'opera prodotta: veniva tutelata l'esclusiva sulla stampa, ma nulla tutelava l'autore del libro stampato.
		Insomma, i primi privilegi non proteggevano i diritti dell'autore(nemmeno esistevano), ma l'ente industriale che produceva l'opera.
		Questo privilegio servì comunque per sottrarre il controllo della stampa alle \textit{corporazioni}, che erano associazioni che regolamentavano e tutelavano l'attività degli artigiani.\\
		Nel 1474 si ebbe una svolta: venne emanato uno statuto che, qualora qualcuno avesse prodotto qualche cosa di meritevole, nuovo e originale avrebbe avuto protezione sulla sua opera per 10 anni. Quindi, per la prima volta, gli interessi non sono più sugli stampatori, ma sugli autori, tutelandone in qualche modo la proprietà intellettuale.
		Il vero cambiamento si ebbe però nel periodo [1517-1537] nel quale le opere presenti caddero nel dominio pubblico. Questo perché fino a questo periodo l'importanza data allo stampatore era ancora dominante su quella data all'autore, generando un mercato di opere dal contenuto fondamentalmente identico ma prodotto da diversi stampatori. A questo punto si capisce che ciò che conta è il contenuto del libro e non la sua forma, dando quindi potere e protezione agli autori.\\
		Col passare del tempo, i privilegi sugli autori andarono a fortificarsi, e vennero addirittura istituite delle protezioni sulle modifiche alle opere(prima un'opera modificata era considerata come un'opera differente dall'originale).
		Storicamente, questo è il periodo rinascimentale, nel quale si vuole dare spazio all'uomo e alla sua creatività.
		La conseguenza di tutto ci fu la nascità del concetto di \textit{nascita immateriale}: le conoscenze di una persona potevano essere importanti.
		
		\paragraph{Chi è Johames di Spira?}: \\ % Domanda incomprensibile
		Si veda la domanda sopra.\\
		Qui la pagina wikipedia:\\
		Giovanni e Vindelino da Spira (Johann e Wendelin von Speyer; Spira, XV secolo – Venezia, XV secolo) sono stati due tipografi tedeschi, attivi nel XV secolo.
		Dopo aver appreso l'arte della stampa a caratteri mobili a Magonza i due fratelli emigrarono in Italia. Giunti a Venezia entro il 1469, impiantarono il primo torchio tipografico nella capitale veneziana.
		
		Avviarono subito la produzione di libri a stampa: i loro primi volumi furono le Epistulae ad familiares di Cicerone e gli Epistularum libri di Plinio il Giovane. Nel 1470, Vindelino terminò l'edizione del De civitate Dei di Sant'Agostino iniziata dal fratello, morto prematuramente. In seguito si diede alla stampa di classici latini (Plauto, Catullo, Marziale, Tacito) e opere liturgiche.
		La loro opera più conosciuta fu la Bibbia in volgare italiano di Nicolò Malermi (1471), la prima traduzione italiana a stampa della Bibbia.
		
		\paragraph{Qual è uno dei grandi cambiamenti dell'editto di Anne?}
		Con l'editto di Anne (1710) si ha, per la prima volta, che il copyright viene visto come una stimolazione alla cultura, protegendo le 
		creazioni intangibili e non più le copie come precedentemente. Le licenze passano dagli stampatori agli autori dei libri. L'autore ha 14 anni di protezione con la possibilità di estendere per altri 14 anni tale protezione. In questo modo, al massimo, un autore poteva proteggere una copia per 28 anni e, al termine, l'opera diventava di dominio pubblico.
	
	\section{Nascita del progetto GNU}

		\paragraph{Chi è Richard Stallman?}: \\
		Nato nel 1953, Stallman è uno dei principali esponenti del movimento del software libero.
		Megli anni '70, Stallman entra a far parte dell'IBM New York Scientific Center, un gruppo di ricerca nell'ambito della computer science. Stallman passò qui l'estate dopo il diploma di scuola superiore sviluppando programmi in Fortran per il calcolo numerico. Al tempo non era ancora un informatico, ovviamente, ma se ne voleva interessare.\\
		Nel 1971 entra ad Harvard, ma si interessa di più al MIT di Boston. Infatti Nel giugno del 1971 al primo anno da studente all'università Harvard, Stallman diventò un programmatore al laboratorio IA (Intelligenza Artificiale) del MIT, assunto da Russ Noftsker come programmatore di sistema. Da qui, prese parte alla comunità degli hacker, lavorando insieme a Richard Greenblatt e Bill Gosper. \\
		Nel 1980, al MIT c'era una stampante(Xerox 9700) al servizio degli utenti. Essa però non aveva un buon driver e alcuni messaggi d'errore non erano comunicati. In particolare, quando essa si inceppava, l'unico modo per scoprirlo era andare a verificare di persona, perdendo del tempo inutilmente. Essendo programmatore di sistema, Stallman voleva fixare questo problema, ma per farlo doveva avere a disposizione il codice sorgente della stampante in questione. Stallman fece richiesta, ma gli venne negato l'accesso al codice sorgente, impedendogli di fixare il bug.\\
		Questo fu un fatto che spinse successivamente Stallman a fondare il progetto GNU(GNU is Not Unix).
		
		\paragraph{Dove aveva cominciato a lavorare prima del MIT?}
		vedi domanda sopra
		\paragraph{Com'è che Stallman entra nella comunità del MIT?}
		vedi domanda sopra
		\paragraph{Qual è il lavoro di Stallman?}
		credo ci si riferisca quando è al MIT, vedi domanda sopra(programmatore di sistema)
		\paragraph{Su cosa comincerà a lavorare Stallman nel MIT?}
		vedi domanda sopra
		\paragraph{Che cos'è TECO?}: \\
		Text Editor and COrrector, è in qualche modo l'antenato di Emacs.\\
		Esso è uno dei primissimo programmi per editing di testo usabili tramite un monitor(e non più tramite telescrivente). Ad essere precisi, TECO è simile a un linguaggio di programmazione interpretato, mirato alla manipolazione del testo. Praticamente ogni carattere è un comando (una sequenza di uno o due caratteri rimpiazza le usuali parole chiave di linguaggi più verbosi) e quindi ogni stringa di caratteri è un programma TECO, anche se non necessariamente un programma utile. È sempre stato considerato molto complicato e, proprio per questo motivo, Stallman decise di cercare un qualche altro editor. \\
		Non trovando risultati soddisfacenti, Stallman scrisse una macro per TECO che permetteva di:
		\begin{itemize}
			\item editare il testo real time(prima non era possibile);
			\item random access editing(non ho ben capito cosa sia, credo sia l'opportunità di modificare qualsiasi punto del testo in tempo costante \textit{O}(1));
			\item aggiungere altre macro.
		\end{itemize}
		Cominciarono a nascere diverse macro per TECO, col fine di migliorarlo.
		Ad un certo punto, però, cominciarono a diventare troppe e disordinate. Guy Steele ebbe in seguito l'idea di fare un po' di ordine, cosa che poi Stallman prese a cuore e continuò. Le macro vennero tra loro omogeneizzate e un disordine del genere non sarebbe più dovuto succedere in futuro. \textbf{da qui parte la storia di Emacs...}
	
		\paragraph{Sotto quale licenza era TECO?}:\\
		non ho trovato nulla per ora. Sicuramente non Emacs License né GNU GPL v1, apparse solo successivamente.
		Sulla documentazione TECO disponibile online si legge:\\

			Copyright (C) 1979, 1985 TECO SIG\\
               General permission to copy or modify, but not\\
               for  profit, is hereby granted, provided that\\
               the above copyright notice  is  included  and\\
               reference  made to the fact that reproduction\\
               privileges were granted by the TECO SIG.\\
		
		
	       
		Immagino sia semplicemente una domanda trabocchetto.
	
		\paragraph{Che cos'è Emacs?} %GNU HURD ???
		Emacs è un editor di testo creato da Stallman. Il precedente editor di testo solitamente era utilizzato TECO. Emacs nasce come un insieme di MACRO per TECO per il miglioramento del display e il realtime random access editing. Poi divenne un editor di testo a sè stante. Inizialmente Emacs era distribuito con una clausola che imponeva che ogni modifica dovessere essere inviata allo sviluppatore principale, in modo tale che, se fosse stata una buona idea, potesse essere resa disponibile a tutti. Ciò permise di creare una comunità attorno a Emacs ma, dall'altro lato, ridusse la libertà di sviluppo.

		\paragraph{Che cosa permetteva di fare Emacs?}
		Emacs, all'inizio, permetteva l'accesso random ai file e permetteva di vedere il testo che veniva inserito in real time.

		\paragraph{Sotto quale licenza venne distribuito Emacs?}
		Emacs viene rilasciato con licenza \textit{GNU Emavs License} o originariamente \textit{Emacs Commune}.
		Emacs è stato rilasciato inizialmente con l'obbligo di inviare allo sviluppatore principale ogni modifica in modo tale che, se fosse stata una buona idea, potesse essere resa disponibile a tutti. Ciò permise di creare una comunità attorno a Emacs ma, dall'altro lato, ridusse la libertà di sviluppo.
		
		\paragraph{Che relazione c'è tra TECO e Emacs?}
		Emacs nasce come un insieme di MACRO per TECO per il miglioramento del display e il realtime random access editing. Poi divenne un editor di testo a sè stante.
		
		\paragraph{Durante la fine del periodo hacker (anni '70), che cosa fa crollare tutto? Chi e perché si organizza uno sciopero?}
		Durante la fine degli anni '70 inizia la fine del periodo hacker. Una delle prime fratture si creò quando al MIT vennero messe le password per l'accesso ai computer dei laboratori. Stallman cerca di convincere le altre persone a limitarne l’utilizzo, e a permettere agli altri utenti di utilizzare i file di tutti. Ciò porta l’intervento del ministero della Difesa, che obbliga l'utilizzo di password. Ciò aliena la comunità vicino a Stallman, anche a causa dell’introduzione dello sciopero del software: Stallman non vuole concedere le ultime versioni di Emacs al laboratorio dell'MIT finchè non venissero tolti tutti i sistemi di sicurezza. \\
		Successivamente, negli anni '70-'80 ha una frammentaione della cultura hacker, causata dal fatto che gli Hacker originari abbandonano il MIT per lavorare o aprire la propria azienda. Si ha un cambiamento dei visitatori al MIT, e con essi cominciano a essere presenti i primi programmi protetti da copyright nel laboratorio di intelligenza artificiale. \\
		La nascita della lisp machine causa la crisi finale. Dall’idea avuta da Greenblat, viene costruita una macchina concepita per funzionare in sintonia con Lisp, ed ebbe un buon successo. Con i copiosi fondi del progetto, al MIT vennero prodotte 32 macchine, che si voleva far comunicare in rete per favorirne la condivisione. Ciò porta Greenblat all’idea di creare un’azienda “hacker friendly” per la produzione di lisp machine. Greenblat si scontrò con Russel Noftsker, che propone, invece, di creare un’azienda “per azioni” e renderla prettamente commerciale. Viene raggiunto un accordo tra i due: a Greenblat viene dato un anno di tempo per creare un’azienda per la vendita di lisp machines. Allo scadere dell'anno, se Greenblat non cci fosse riuscito allora Nofsker avrebbe creato l'azienda. Greenblat ci ci riesce e crea la LMI ma nonostante ciò viene creata un’altra azienda da Nofsker, la Symbolics. Ciò causò causo lo svuotamento del MIT poichè le due aziende attinsero dal MIT per cercare sviluppatori. Inizialmente le due aziende utilizzavano un sistema operativo comune e concessero al MIT di utilizzarlo, ma nel 1982 la Symbolics introdusse delle modifiche al tala sistema operativo e le rese proprietarie. Ciò causo l'ira di Stallman che si mise a replicare le nuove funzionalità per donarle alla LMI. Ciò non fece altro, però, che indebolire la Symbolics. In tutto questo la comunità hacker si indebolisce sempre più, orfana dei fondatori del movimento e incapace di mantenere l'ITS all'interno dei laboratori del MIT. Il crollo definitivo si ha quando nei laboratori l'ITS, diventato ormai obsoleto e poco sicuro, viene sostituito da Tweenex, un software proprietario.

		\paragraph{Chi ha inventato le Lisp machine? Cosa sono? Quali aziende le hanno inventate? Perché queste aziende hanno avuto un impatto sulla comunità hacker?}
		{per una risposta più completa vedi domanda precedente}
		Le Lisp machine sono state inventate da Greenblat, uno dei fondatori del movimento hacker. Le Lisp machine sono delle macchine concepite al fine di lavorare in sintonia con il linguaggio di programmazione Lisp. Tali macchine sono poi state rivendute da due aziende, l'LMI di Greenblat e la Symbolics di Nofsker. Ciò porterà alla crisi definitiva della comunità hacker causata da:
		\begin{itemize}
			\item Gli hacker del MIT vengono chiamati a lavorare nelle due aziende;
			\item La Symbolics crea software con modifiche non rilasciate all'altra azienda;
			\item I padri del movimento hacker, non più al MIT, non riescono a sostenere il loro software interno e ciò porta all'utilizzo di software proprietario.
		\end{itemize}

	\section{Linux}
		
		\paragraph{Racconta la nascita di Linux partendo da il sistema UNIX passando per MINIX}: \\
		\textit{...In origine era UNIX...}\\
		All'epoca, UNIX era un sistema operativo affermato. Nella seconda metà degli anni '70 John Lions(sviluppatore australiano) pubblicò "\textit{Lion' commentary on UNIX 6th Edition, with source code}", un'opera che conteneva il sorgente di UNIX 6.0 commentato da Lions. Con l'avvento di UNIX 7.0 vennero però imposte diverse pesanti restrizioni, in particolare l'opera di Lions venne bloccata, aumentarono i costi delle licenze e ne venne limitato l'insegnamento nelle università. Questa non fu una buona mossa, in quanto la forza di UNIX stava nella diffusione nel mondo accademico.\\
		Questo costrinse alcuni docenti ad adottare delle soluzioni per l'insegnamento, uno dei quali fu Andrew S. Tanenbaum.
		Tanenmbaum all'epoca era docente di computer science e aveva sempre insegnato basandosi su UNIX. Decise quindi di creare un sistema operativo tutto suo,  minimale, semplice e a solo scopo didattico: nacque così \textbf{MINIX}, un OS ad approccio microkernel rilasciato sotto licenza permissiva ma non libera.
		Minix aveva però un problema, ovvero non disponeva di un emulatore di terminale.\\
		Linus Torvald(1969) è stato tra i primi utilizzatori di MINIX. Nel 1990 frequenta l'università di Helsinki, comincia a studiare i testi di Tanenbaum e a modificare MINIX in maniera tale da creare un emulatore di terminale. Nel 1991 scrisse un emulatore di terminale in Assembly e in C dando così vita a Linux (al tempo in realtà si chiamava Freax ed era software proprietario). Da lì venne esteso fino a creare un vero e proprio kernel che poteva supportare un OS (Linux 0.0.1). Vista ormai l'estensione raggiunta, Torvalds decise nel 1992 di "staccare" Linux da MINIX e renderlo indipendente, adottando una licenza differente(la GPL).
		
		\textbf{...continua con l'evoluzione e diffusione di Linux...}
		
		
		\paragraph{Come si è evoluto e diffuso Linux?}
		Nel 1990 frequenta l'università di Helsinki, comincia a studiare i testi di Tanenbaum e a modificare MINIX in maniera tale da creare un emulatore di terminale. Tale emulatore può essere considerato come la primissima versione di Linux, poichè prevedeva già due thread, uno per la gestione della tastiera, l'altro per il monitor. Nel 1991, sviluppando ancor di più il terminale e implementando tutte le chiamate POSIX, crea la versione 0.0.1 di Linux. Questa prima versione era rilasciata sotto licenza proprietaria e si chiamava Freax, però il professore che gli permise di caricare tale lavoro sul server gli cambiò nome in Linux. Linus continua successivamente a sviluppare il software ed entra in contrasto anche con Tanenbaum, poichè, infatti, il codice di Minixx era stato rilasciato sotto licenza proprietaria. Nel '92 viene rilasciato Linux 0.12, che si stacca completamente da Minix e adotta come licenza la GPL, con la quale Linus permettè a chiunque di contribuire. Nel 1994 si arriva alla versione 1.0 Linux. \\
		Grazie all'adozione della GPL come licenza, anche varie aziende si interessarono al progetto di Linus come SUSE o RedHat Software Inc. che crearono le proprie distribuzioni. Successivamente naque anche Debian. 
	
	\section{Open source}	
		\paragraph{Come è nato il movimento Open-Source?}: \\
		Il movimento Open Source nasce ovviamente dai principi già affermati ed esistenti del \textit{free software}. Negli anni '90, il movimento del free software non era ben visto nel mondo commerciale/business in quanto le grandi aziende mal vedevano questo filone culturale. Probabilmente anche il termine \textit{free} ha contribuito a ciò, in quanto spesso viene confuso con "gratuito" quando, invece, sta per "libero". \\
		Nel 1997, Eric Raymond(informatico statunitense) scrisse "\textit{The Cathedral and the Bazaar}", cioè un saggio che analizza Linux e la sua architettura, sostenendo che quest'ultimo fosse destinato al collasso e al fallimento. Raymond sosteneva che, nella comunità degli sviluppatori che si dedicavano a Linux, ognuno avrebbe (potuto fare) fatto di testa sua, causando lo "scoppio" di Linux. Il punto è che ciò non accadeva e Raymond, nel suo saggio, ne analizza il motivo.\\
		Il risultato dell'analisi fu una serie di linee guida che, in qualche modo, gli sviluppatori tendevano a seguire.
		Raymond elencò e spiegò quali sono quindi le regole che una comunità di sviluppatori dovrebbero seguire per non far "esplodere/collassare" il loro progetto:
		\begin{itemize}
			\item ogni progetto dev'essere un "prurito" del programmatore;
			\item i buoni programmatori sanno cosa scrivere, i grandi programmatori sanno cosa RIscrivere;
			\item tratta i tuoi utenti come co-sviluppatori, come se il programma fosse stato fatto insieme a loro. Questo porta molte persone a migliorare il software, in quanto esse si sentono parte del progetto.  Inoltre, la varietà di persone porta ad una varietà di soluzioni allo stesso problema con approcci diversi, e la combinazione dei contributi può significare grossi miglioramenti.
			\item distribuisci presto, spesso e presta ascolto ai tuoi utenti, in modo da trovare e risolvere i bachi velocemente.
			\item «Dato un numero sufficiente di occhi, tutti i bug vengono a galla», Eric Raymond. Quindi dato un sufficiente numero di beta tester ogni 
			problema verrà identificato e risolto.
			\item Riconoscere le buone idee degli utenti è importante 
			come averne di proprie.
		\end{itemize}
		Il saggio di Raymond riscosse molto successo(rendendo l'autore famoso) non solo tra gli sviluppatori, ma anche tra alcune grandi aziende, uno tra esse Netscape.
		Ispirata dal saggio di Raymond, nel 1998 Netscape rilasciò il codice sorgente del proprio browser.
		In realtà, Netscape decise di pubblicare il codice anche per un altro motivo: in quel periodo, Microsoft stava sviluppando un suo browser e a Netscape pensavano che sarebbe potuto diventare dominante nel mercato, destinando Netscape alla deriva. Proprio per questo, rilasciarono il sorgente sotto licenza libera. Su questa decisione, ebbe un'importante influenza Raymond. \\
		Per i motivi elencati inizialmente, c'era però il rischio che il termine "free software" non assumesse la giusta importanza(e di conseguenza Netscape che aveva dichiarato il suo browser free software). Venne coniato così, all'interno di Netscape, il termine "\textit{Open Source}". Di conseguenza, nel 1998 Eric Raymond e Bruce Perens fondarono l'OSI(Open Source Initiative), avente lo scopo di conciliare la filosofia del free software con le regole di mercato.
		
		
		\paragraph{Chi scrive "La cattedrale e il bazaar"? Di cosa parla? In quali anni?}: \\
		Vedi domanda precedente. \\
		\textbf{--- breve spiegazione del titolo del saggio ---} \\
		Nel modello a cattedrale il programma viene realizzato da un numero limitato di "esperti" che provvedono a scrivere il codice in quasi totale isolamento. Il progetto ha una suddivisione gerarchica molto stretta e ogni sviluppatore si preoccupa della sua piccola parte di codice. Le revisioni si susseguono con relativa lentezza e gli sviluppatori cercano di distribuire programmi il più possibile completi e senza bug. Il programma Emacs, il GCC e molti altri programmi si basano su questo modello di sviluppo.\\
		Nel modello a bazaar il codice sorgente della revisione in sviluppo è disponibile liberamente, gli utenti possono interagire con gli sviluppatori e se ne hanno le capacità possono modificare e integrare il codice. Lo sviluppo è decentralizzato e non esiste una rigida suddivisione dei compiti, un programmatore di buona volontà può modificare e integrare qualsiasi parte del codice. In sostanza lo sviluppo è molto più diluito e libero, da qui il nome di modello a bazaar. Il kernel Linux e molti programmi utilizzano questo nuovo modello di sviluppo associativo.
		\\
		\textbf{--- fine ---}
		
		\paragraph{Quale grande evento ha fatto si che Raymond diventasse famoso?\\}
		Sia per il saggio pubblicato che per la decisione di rilasciare il sorgente di Netscape(principalmente per quest'ultimo).
		
		\paragraph{A quali scopi Netscape rilascia il codice sorgente?}
		: \\ vedi domanda precedente
		\paragraph{Cosa relaziona il rilascio del codice sorgente di Netscape a Raymond?}
		 vedi domanda precedente
		
		\paragraph{Chi è Bruce Perens?}
		Bruce perens è insieme a Raymond uno dei fondatori della Open Source 
		Initiative (OSI). Perens fu il primo project leader di Debian, ovvero
		il coordinatore dello sviluppo di Debian, e redasse 
		le Debian Free Software Guidelines (DFSG) che con le dovute modifiche
		(p.es. rimozione di referenze specifiche a Debian) diverranno la 
		Open Source Definition (OSD).

		\paragraph{Parlami della Open Source Definition}
		La open source Definition è un insieme di linee guida chiare per 
		considerare una licenza come Open Source secondo la Open Source 
		Initiative (OSI). Derivate dalle Debian Software Guidelines (DFSG)
		ma rese più generali per includere molto software libero sviluppato 
		prima della fondazione di OSI (per esempio TeX).
		
	\section{Creative Commons}
	
		\paragraph{Come nasce il primo movimento di Creative Commons?}
			Nata ufficialmente nel 2001 grazie al prof. Lawrence Lessig, il movimento CC nasce come associazione non lucrativa per diffondere la politica di copyleft anche nelle opere non informatiche (nascevano infatti nuovi formati digitali per la trasmissione di opere artistiche). 
			
			Tutto è stato anticipato dalla licenza GNU Free Documentation License (GFDL) che mirava a proteggere la documentazione software, cosa che non avveniva con la sola GPL. Essa garantisce libertà di modifica e regola la distribuzione per grandi quantità imponendo restrizioni sulle redistribuzioni senza modifica e con.
			
			Nel 1990 la legge del diritto d'autore prendeva sempre più piede, il professore di legge Lawrence Lessig aspira ad essere un attivista e pubblico intellettuale e visto che in quegli anni il Web stava esplodendo si fissa a combattere il diritto d'autore sempre più restrittivo e a valorizzare il dominio pubblico come progresso per la cultura umana. 
			Da un altra parte Eric Eldred, informatico, il quale aveva lavorato anche per software spaziali, mantiene un sito online dove distribuisce i libri non più coperti da diritti d'autore (in particolare la letteratura classica americana). Nel 1998 però viene approvato l'Extension Copyright Act che prolunga la durata del copyright per 20 anni per le opere sotto diritti d'autore. Questo significava per Eldred perdere tutta la letteratura che negli anni a venire sarebbe divenuta di dominio pubblico (prima il copyright durava 75 anni per le imprese e 50 dalla morte degli autori, ora invece era 95 e 70). L'estensione è avvenuta soprattutto dalle spinte della Disney la quale avrebbe perso i diritti sul personaggio di Mickey Mouse (1928, quindi nel 2003 sarebbe divenuto pubblico).
			Nel 1998 Lessig e Eldred si incontrano ed entrambi sono d'accordo nello sfidare l'atto legale e lo combattono attraverso la legge stessa dichiarandolo anticostituzionale, infatti esso limitava la diffusione della cultura, per salvare infatti il 2\% delle opere ancora utili alle imprese il 98\% delle opere ormai non avevano più un business di vendita perciò sarebbero rimaste non usufruibili.
			Durante la lotta legale contro il Copyright Extension Act, nascono online comunità per confrontarsi sull'argomento, primo fra tutti il sito openlaw dove si discuteva sul come diffondere le informazione nel mondo. Nel 1996 si ha la pubblicazione della Dichiarazione d'Indipendenza del cyberspazio. Si rivela quindi necessario che il materiale diffuso online e liberamente richiedesse una regolamentazione. Da qui Lessig nel 2001 fonda il movimento CC, un'associazione non lucrativa per regolare la diffusione delle opere che volevano essere distribuite al pubblico dominio.  Intanto la denuncia contro il CEA continua fino alla sconfitta di Lessig-Eldred nel 2003 alla corte suprema.
		
		\paragraph{Qualcuno ha guidato il movimento Open Content? Chi sono state le prime persone che si sono battute per questi diritti?}
		Eric Eldred che voleva combattere la legge sul copyright secondo lui troppo restrittiva e Lawrence Lessig che voleva proteggere il bene comune tramite la legge. Vedi risposta precedente.
		
		
	\section{Licenze}
	
		\subsection{GPL v2}
		
			\paragraph{Illustrare tutti gli obblighi che una persona deve soddisfare per poter distribuire solo i binari (nel testo)}
			Tali obblighi sono descritti nella sezione 3 della licenza. Essi descrivono cosa deve accompagnare il codice oggetto (program) o l'eseguibile. Le opzioni sono esclusive.
			\begin{itemize}
				\item Deve essere accompagnato dal relativo codice sorgente anch'esso distribuito sotto le regole di questa licenza; oppure
				\item Deve essere accompagnato da un'offerta scritta, valida per almeno 3 anni, di fornire il codice sorgente a chiunque ne faccia richiesta in cambio di un costo non superiore al costo di trasferimento fisico, esso va fornito secondo le regole di questa licenza; oppure
				\item Deve essere accompagnato dalle informazioni che sono state ricevute riguardo alla possibilità di ottenere il codice sorgente solo nei casi:
				\begin{itemize}
					\item di distribuzioni non commerciali; e
					\item in cui il programma è stato ottenuto sotto forma di codice oggetto o eseguibile in accordo al precedente comma.
				\end{itemize}
			\end{itemize}
		
			\paragraph{Nel caso di sorgenti modificati quali sono gli obblighi? (Illustrare nel testo)}
				Tali obblighi sono descritti nella sezione 2 della licenza. Essi regolano la modifica o la copia del programma o porzioni di esso.
				\begin{itemize}
					\item Indicare chiaramente nei file se si trattano di copie modifiche e indicare la data della modifica; e
					\item ogni opera distribuita che deriva dal programma o da parti di esso deve essere concessa nella sua interezza secondo i termini di questa licenza; e
					\item se il programma modificato legge comandi interattivamente quando viene eseguito, esso all'inizio deve mostrare un messaggio contenente una nota di copyright e di garanzia (anche assente se lo è). Il messaggio deve inoltre specificare che chiunque intenda ridistribuire l'opera secondo questa licenza.
				\end{itemize}
		
			\paragraph{Cosa fare nel caso dovessi passare la libreria a una terza persona? Ci sono limitazioni?}
				Si applicano le stesse regole per qualsiasi software regolato da GPL. Ciò significa che un software che fa uso di tale libreria dovrà essere distribuito sotto licenza GPL.
				Nonostante questo l'uso della GPL per una libreria è scoraggiato, la GPL v2 infatti consiglia l'uso della licenza meno restrittiva Lesser GNU Public License (LGPL) la quale permette l'uso di librerie software sotto LGPL per software proprietari a patto che queste non siano modificate ma solo "collegate". Un software che fa uso di software regolato da LGPL può essere licenziato con la GPL.
				La licenza è nata per poter far sì che anche il software proprietario usasse la libreria C nel sistema GNU. Se tale libreria fosse stata regolamentata dalla GPL avrebbe scoraggiato l'uso del sistema GNU anche per chi avesse voluto sviluppare software proprietario. Oltre a questo si fa notare che la GPL racchiude molta ambiguità tra i termini \textit{opera derivata} e \textit{opera collettiva}, la LGPL nonostante il tentativo non li risolve. (fonte: Open Source Licensing - Lawrence Rosen)
				
			
			\paragraph{Come rendere disponibile la consultazione dei codici sorgenti di binari con licenza GPL v2?} % 2 possibilità
			Allegando il codice sorgente ai binari stessi oppure allegando una offerta scritta valida per 3 anni la quale afferma che è possibile distribuire il codice sorgente previa richiesta con il costo non superiore al trasferimento fisico del codice sorgente (il costo del CD/DVD e del pacco postale nel caso fosse inviato fisicamente, con Internet è pari a zero).
			
			\paragraph{Cosa contiene la clausola "Libertà o morte"?}
				La clausola 7 specifica che qualora ci siano restrizioni di qualsiasi tipo sulla distribuzione nei termini indicati dalla licenza, il software non può essere distribuito. La clausola serve per scoraggiare l'imposizione di regole legali nella distribuzione del software.
		
		\subsection{GPL v3}
			
			\paragraph{Illustrare le caratteristiche principali}
			La GPL v3 presenta molte differenze con la GPL v2, poichè, nel corso degli anni si sono presentati una serie di problemi che quando è stata creata la GPL v2 non erano presenti.
			\begin{itemize}
				\item \textbf{Tratta il tema dei brevetti software.} In Europa non è possibile richiedere brevetti per il software come su teorie matematiche. Nel corso degli anni la legge è stata modificata è permette quindi di brevettare software se questo è previsto all'interno di un prodotto completo. Ciò portò allo sfruttamento di tale norma e venne aggirato il vincolo dei brevetti software brevettando macchine con il solo scopo di brevettarno il software all'interno. Nel software libero è possibile ccomunque rilasciare software protetto da brevetti a patto che tale codice venga rialsciato insieme ai brevetti.
				\item \textbf{Tratta il tema della Tivoizzazione.} Il Tivò era un apparecchio per registrare le trasmissioni televisive all'interno dell'hard disk della televisione. Tale apparecchio utilizzava Linux ma aveva un bootloader che non permetteva delle modifiche alla distribuzione utilizzata. La GPL v3 obbliga a dare all'utente tutto il materiale necessario a far fuzionare il prodotto se un utente vuole modificare il software.
				\item \textbf{Tratta il tema della delle protezioni digitali.} La GPL v3 afferma che rilasciando il software sotto questa licenza si prende atto di non considerare le proprie modifiche come protezioni digitali.
				\item \textbf{Internalizzazione.} Si svincola dalla legislazione americana, definendo ogni termine (per evitare interpretazioni ambigue) e mantiene come lingua ufficiale l'inglese. In questo modo diventa compatibile con un numero maggiore di sistemi giuridici.
				\item \textbf{Regola il lavoro su commissione.} Permettendo di limitare la modifica e/o la ridistribuzione in caso di lavori su commissione.
				\item \textbf{Permette la distribuzione dei codici via internet e peer2peer.}
				\item \textbf{Permette una serie di restrizioni.} ciò rende la licenza compatibile con l'Apache v2.

			\end{itemize}
		
		\subsection{BSD}
		
			\paragraph{Illustrare brevemente 2a, 3a e 4a clausola}
				La BSD versione 1.0 è costituita da 4 clausole, in seguito usciranno versioni senza la 4a clausola e poi senza la 3a e 4a.
				\begin{description}
					\item[2 clausola] indica che qualsiasi ridistribuzione dei file binari deve riportare il testo della licenza con i tre elementi: nota di copyright, lista delle condizioni, avvisi di garanzia e responsabilità;
					\item[3 clausola] il materiale pubblicitario che interessa il software sotto la licenza deve mostrare il riconoscimento:  "Questo prodotto include software sviluppati dalla <organizzazione>." (la seguente clausola è stata poi tolta perché era una sorta di pubblicità gratuita e interferiva con la libertà del software derivato)
					\item[4 clausola] il nome dell'<organizzazione> e dei collaboratori non può essere usato per scopi commerciali.
				\end{description}
			
		\subsection{Apache}
		
			\paragraph{Spiegare la clausola di rinomina}
				La clausola di rinomina fa sì che la licenza conceda l'uso del software ma non del marchio del software e dell'organizzazione.
				Nel caso Apache, la licenza protegge il nome Apache e Apache Software Foundation per scopi pubblicitari senza il permesso e per l'uso del nome stesso in progetti derivati senza il permesso. Al contrario della BSD il marchio è protetto.	

	\section{Date importanti}
	
	\begin{description}
		\item[1700:] anni degli editti a Venezia (proto-copyright)
		\item[1710:] editto di Ann
		\item[1976:] Copyright UA
		\item[1983:] Manifesto GNU di Stallman
		\item[1991:] Prima versione di Linux
	\end{description}
	
	% --- STRUMENTI OPEN SOURCE --- %
	
	\section{SVN}
	
	\paragraph{Creare un repository SVN}: \\
	Subversion è un sistema di versionamento centralizzato, ergo è necessario che il repository alloggi su un server. Noi lavoreremo in locale, ergo il server è \textit{localhost}. In realtà è possibile avere un repository con percorso locale, il che è l'ideale per imparare a utilizzare Subversion.\\
	Si dovrà dare il comando:
	\begin{verbatim}
	svnadmin create <path>
	\end{verbatim}
	
	\paragraph{Creare una working copy con checkout}:\\
	Il comando \textit{checkout} serve per crearsi in locale una copia del progetto. È lo stesso concetto ci \textit{clone} di git.
	\begin{verbatim}
	svn checkout <url> <path>
	\end{verbatim}
	dove \textit{<url>} è la posizione del repository, mentre \textit{<path>} è il percorso LOCALE nel quale vogliamo copiarlo.
	Rispetto a clone di git, con checkout di svn è possibile copiare delle cartelle o singoli file del repo, semplicemente specificandone il percorso nel server.
	
	\paragraph{Mostrare un ciclo modifica -> commit -> update}
	\begin{itemize}
		\item modifica: consiste nella modifica di un file;
		\item commit: questo comando invia le modifiche della working copy al server.\begin{verbatim}
		svn commit --message "Messaggio del commit" [<path>]
		\end{verbatim}
		Il messaggio è obbligatorio e path è il percorso LOCALE dal quale caricare le modifiche. Se path non è specificato, allora assume il valore di default, ovvero la cartella corrente.
		\item update: questo comando aggiorna la working copy con la copia aggiornata del repository.
		\begin{verbatim}
		svn update
		\end{verbatim}
	\end{itemize}
	
	\paragraph{Mostrare l'uso dei comandi svn status e revert <file>}
	\begin{itemize}
		\item status: stampa lo stato dei file e cartelle della working copy.
		\begin{verbatim}
		svn status
		\end{verbatim}
		Per mostrane l'uso si può modificare un file, NON committarlo e poi dare status.
		\item revert: vengono annullate le ultime modifiche in locale.
		\begin{verbatim}
		svn revert
		\end{verbatim}
		Per mostrarne l'uso si può fare l'update, fare delle modifiche ai file e poi dare revert.
	\end{itemize}
	\paragraph{Quali sono le opzioni di svn status?}: \\
	In \textbf{grassetto} sono evidenziati quelli importanti.
	\begin{itemize}
		
		
		\item - -changelist (- -cl) ARG, questo comando dice su quali file operare. ARG è la lista dei file
		\item - -depth ARG, questo comando dice il limite dello scope dell'operazione.
		\item - -ignore-externals
		\item - -incremental
		\item - -no-ignore, vengono considerati anche i file che svn normalmente ignora(settati con la proprietà svn:ignore. Vedere domanda "Ignorare tutti i file ...")
		\item \textbf{- -quiet (-q)}, stampa un riassunto delle informazioni sugli elementi localmente modificati
		\item \textbf{- -show-updates (-u)}, stampa le revisioni di lavoro e informazioni su eventuali problemi di out-of-date con il server(out of date:=alterato localmente ma non aggiornato sul server).
		\item \textbf{- -verbose (-v)}, stampa dettagliatamente tutte le informazioni di revisione su ogni elemento locale.
		\item \textbf{- -xml}, stampa l'output in formato XML 
		
	\end{itemize}
	Esempio: vedere quali file sono out-of-date nella working copy
	\begin{verbatim}
	svn status -u wc
	\end{verbatim}
	dove wc è la working copy.
	\paragraph{Eseguire il backup del repository}: \\
	Si può fare in diversi modi:
	\begin{itemize}
		\item hotcopy con svnadmin: viene creata una copia di backup comprendente i file di configurazione, quelli del DB e gli hooks(programmi attivati da certi eventi sul repository). \begin{verbatim}
		svnadmin hotcopy <path_repo> <new_path_repo>
		\end{verbatim}
		\item dump con svnadmin: scrive(dump) il contenuto del filesystem in un "dump file" portable format tramite uno stream di output stdout.\begin{verbatim}
		svnadmin dump <path> > full.dump
		\end{verbatim}
		
		\item migrare un progetto ad un nuovo repo...?
		
	\end{itemize}
	\paragraph{Cancellare un repository}: \\
	Lo si può fare con i comandi non specifici di svn.
	\begin{verbatim}
	rm -rf <path>
	\end{verbatim}
	
	\paragraph{Ripristinare il repository}
	\begin{itemize}
		\item load con svnadmin: legge da uno stream di input stdin il dump di un repository
		\begin{verbatim}
		svnadmin load <path> < repos_backup.dump
		\end{verbatim}
	\end{itemize}

	\paragraph{Mostrare come riconoscere se un file è stato modificato}
	Basta dare il comando 
	\begin{verbatim}
		svn status
	\end{verbatim}

	\paragraph{Eseguire un commit ignorando alcuni file}: \\
	Si possono commitare file singoli. Poniamo di voler committare tutti i file .html e non quelli .php. Il comando sarà:
	\begin{verbatim}
	svn commit -m "messaggio di commit" *.html
	\end{verbatim}
	\paragraph{Ignorare tutti i file all'interno di una precisa cartella}: \\
	Bisogna settare la proprietà \textit{ignore}, in questo modo:
	\begin{verbatim}
	svn propset svn:ignore <path> .
	\end{verbatim}
	dove path è il percorso della cartella.\\
	Lo si può fare in modo analogo con singoli file.
	Se invece vogliamo ignorare più tipi di file(quello che in git è fatto da \textit{.gitignore}) w/o più cartelle, si crea un file contenente tutte le estensioni dei file e i path alle cartelle che vogliamo ignorare, ad esempio
	\begin{verbatim}
	folder_eseguibili
	docs/ndp
	.o 
	.out
	.exe
	.toc 
	\end{verbatim}
	nominiamo questo file \textit{svnignore.txt} e diamo il comando
	\begin{verbatim}
	svn propset svn:ignore -F svnignore.txt
	\end{verbatim}

	\paragraph{Ottenere conflitti tra due utenti}
	Per prima cosa creiamo un repository
	\begin{verbatim}
	svnadmin create ProvaRepo
	\end{verbatim}
	Ora cloniamo il file in una cartella che simula un utente A e creiamo
	il file \verb|prova.txt|:
	\begin{verbatim}
	svn co file:///path/to/the/repo/ProvaRepo A/
	#In alternativa se il file è nella stessa cartella: svn co file://$PWD/ProvaRepo 
	cd A/
	echo "A" > prova.txt 
	svn add prova.txt
	svn commit -m "committed file prova.txt"
	cd ..
	\end{verbatim}
	Ora creiamo un altro collegamento a quel repository, simulando l'utente B:
	\begin{verbatim}
	svn co file:///path/to/the/repo/ProvaRepo B/
	cd B/
	echo "B" >> prova.txt
	svn commit -m "changed file prova.txt"
	cd ..
	\end{verbatim}
	Ora torniamo nella cartella dell'utente A e modifichiamo il file prova e 
	committiamo i cambiamenti, per esempio:
	\begin{verbatim}
	cd A/
	echo "A" >> prova.txt
	svn commit -m "changed file prova.txt"
	\end{verbatim}
	Otterremo un conflitto

	\paragraph{Risolvere manualmente un conflitto e committare}
	Creiamo un conflitto come nella domanda sopra. Quando viene dato il comando 
	\begin{verbatim}
	svn update
	\end{verbatim}
	E scegliamo come opzione ``p'' (postpone). Ora possiamo aprire il file dove c'è il conflitto con un editor di testo e risolvere il conflitto. Fatto questo diamo il comando
	\begin{verbatim}
	svn resolved prova.txt
	\end{verbatim}
	Per comunicare al sistema che il conflitto è stato risolto.

	\paragraph{Cancellare le modifiche dopo averle committate}
	Dare il comando 
	\begin{verbatim}
	svn status -v -u
	\end{verbatim}
	Che permette di vedere il numero della revisione a cui si è arrivati e successivamente dare il comando
	\begin{verbatim}
	svn merge -r n_att:n_prev
	\end{verbatim}
	Dove n\_att è il numero della revisione attiale mentre n\_prev è il numero della revisione precedente.
	
	\paragraph{Caricare il repository in Internet}
	Creiamo un repository e mettiamo dentro un file
	\begin{verbatim}
	svnadmin create /home/zanna/svn/repos/test
	svn checkout file:///home/zanna/svn/repos/test
	cd test
	echo 'Hello, World!' > hello.txt
	svn add hello.txt
	svn commit -m "Added a 'hello world' text file."
	\end{verbatim}
	Ora andiamo a modificare il file conf/passwd per settare la password di un utente
	\begin{verbatim}
	gedit /home/zanna/svn/repos/test/conf/passwd
	[users] 
	zanna = password
	\end{verbatim}
	Ora andiamo a modificare il file conf/svnserve.conf impostando il file precedentemente modificato come file per le password e impostando i permessi per gli utenti senza password
	\begin{verbatim}
	gedit /home/zanna/svn/repos/test/conf/svnserve.conf
	[general] 
	anon-access = none 
	password-db = passwd 
	realm = prova
	\end{verbatim}
	Adesso possiamo far partire il servizio
	\begin{verbatim}
	sudo svnserve -d --foreground -r /home/zanna/svn/repos
	\end{verbatim}
	Ora quindi è possibile accedere al repository inserendo l'IP della macchina in cui si trova. Nel  nostro caso è nella stessa macchina quindi l'indirizzo è quello del localhost
	\begin{verbatim}
	svn checkout svn://127.0.0.1/test
	\end{verbatim}
	
	\paragraph{Accedere al repository da remoto}
	Penso sia come la domanda sopra
	
	\paragraph{Creare dei permessi per l'accesso al repository}
	Per settare i permessi da remoto si deve:
	\begin{itemize}
		\item per gli utenti senza una password è necessario accedere al file conf/svnserve.conf e modificare la riga in cui è presente ``anon-access = ''. Nel caso in cui non sia presente è necessario aggiungerla;
		\item per gli utenti con password è necessario modificare il file conf/authz \textbf{NON SO COME SI SETTINO DI PRECISO}.
	\end{itemize}

	\paragraph{Copiare il contenuto di una cartella tranne del repository in una cartella branches/stable. Inserire un messaggio}

	\paragraph{Illustrare le conseguenze dell'uso del comando svn update -r}
	Permette far tornare la working copy alla revisione specificata.

	\paragraph{Mostrare come risolvere l'errore "out-of-date"}

	\paragraph{Illustrare come è possibile evitare i conflitti tra utenti differenti su di un file}
	È possibile utilizzare
	\begin{verbatim}
	svn lock nome_file
	\end{verbatim}
	e per sbloccare 
	\begin{verbatim}
	svn unlock nome_file
	\end{verbatim}
	Lock e unlock sono 2 dei pochi comandi di cui \textbf{NON} è necessario fare il push perchè siano propagati al repository.
	È comunque possibile rubare il lock di qualcun altro con
	\begin{verbatim}
	svn unlock --force nome_file
	\end{verbatim}

	\paragraph{Se dovessi utilizzare un hook per impedire ad un utente di fare un commit, che hook useresti?} O un hook pre-commit oppure un hook start-commit.

	\paragraph{Lista riassuntiva dei comandi\\}
	\begin{verbatim}
	svnadmin create /usr/local/svn/newrepo
	\end{verbatim}
	Crea un repository nuovo in quel path.\\-----
	\begin{verbatim}
	svn mkdir file:///usr/local/svn/newrepo -m "messaggio"
	\end{verbatim}
	Aggiungo la cartella sotto controllo di versione. Serve il messaggio(-m) in quanto questo comando fa un commit.\\-----
	\begin{verbatim}
	svnserve -d
	\end{verbatim}
	Lancio il thread demone lato server per rendere disponibile online il repo.\\-----
	\begin{verbatim}
	svn import <path> <URL> -m "messaggio"
	\end{verbatim} 
	Committa il file <path> non sotto controllo di versionamento nel repository <URL>. Serve il messaggio perchè è, appunto, un commit.\\
	Dopo questo comando, è obbligatorio fare un checkout, in quanto il repo nel server avrà ora una struttura differente rispetto al working copy locale: abbiamo dei file nel repo remoto che, in locale, non sono sotto controllo di versionamento.\\-----
	\begin{verbatim}
	svn checkout <URL> <path> 
	\end{verbatim}
	Crea una copia di lavoro dal repo URL in locale nel percorso path.\\-----
	\begin{verbatim}
	svn ls
	\end{verbatim}
	Elenca i file contenuti nella directory corrente.\\-----
	\begin{verbatim}
	svn commit -m "messaggio"
	\end{verbatim}
	Commita nel server. Il messaggio è obbligatorio.\\-----
	\begin{verbatim}
	svn update
	\end{verbatim}
	Aggiorna la working copy con il repository aggiornato nel server. Questo comando si occupa anche di unire le modifiche locali con quelle nel repository.\\
	Con l'opzione \textit{-i x} si può tornare alla versione \textit{x} di quel repository.\\
	Per ogni file aggiornato, sarà riportata una tabella con le seguenti lettere:
	\begin{itemize}
		\item A aggiunto
		\item D, cancellato
		\item U, aggiornato
		\item C, conflitto da risolvere
		\item G, fuse le modifiche, ovvero merge
		\item E, esistente
		\item R, sostituito
	\end{itemize}
	-----
	\begin{verbatim}
	svn add [path]
	\end{verbatim}
	Aggiunge un elemento alla working copy e, al primo commit, al repo.\\-----
	\begin{verbatim}
	svn delete [path]
	\end{verbatim}
	Analogo al comando precedente.\\-----
	\begin{verbatim}
	svn copy [source] [destination]
	\end{verbatim}
	-----
	\begin{verbatim}
	svn move [source] [destinatione]
	\end{verbatim}
	-----
	\begin{verbatim}
	svn revert [path]
	\end{verbatim}
	Annulla tutte le modifiche locali.\\-----\\
	Dopo aver fatto un \textit{update}, alcuni file potrebbero essere marcati con il flag \textbf{C} (conflitto). È neccessario risolvere il conflitto a mano e, successivamente, dare:
	\begin{verbatim}
	svn resolved file.txt
	\end{verbatim}
	In questo modo diciamo a SVN che i conflitti sono stati risolti ed è quindi posssibile committare. Quindi, infine, dare:
	\begin{verbatim}
	svn commit -m "messaggio"
	\end{verbatim}
	-----\\
	\begin{verbatim}
	svn status
	\end{verbatim}
	Stampa informazioni sui file modificati localmente.
	\\-----
	\begin{verbatim}
	svn diff
	\end{verbatim}
	Stampa le differenze apportate ai file modificati.\\-----
	\begin{verbatim}
	svn log
	\end{verbatim}
	-----
	\begin{verbatim}
	svn cat -r x file.txt 
	\end{verbatim}
	Viene stampato il contenuto del file alla versione \textit{x}. Senza l'opzione \textit{-r x}, viene stampato il contenuto attuale.\\-----
	\begin{verbatim}
	svn list
	\end{verbatim}
	Mostra i file nel repository, ma senza che essi vengano scaricati.\\-----
	\begin{verbatim}
	svn cleanup
	\end{verbatim}
	-----
	\begin{verbatim}
	svn merge
	\end{verbatim}
	Applica le differenze tra due\\-----
	\begin{verbatim}
	svn export URL [path]
	\end{verbatim}
	Copia il repository URL senza i file di SVN, ovvero otteniamo il repo non sotto controllo di versione.\\-----
	\begin{verbatim}
	svn export path1 path2
	\end{verbatim}
	Come il precedente, ma tutto in locale.\\-----
	\begin{verbatim}
	svnadmin dump <path> > file.dump
	\end{verbatim}
	Backup del repo in <path> nel file.dump.\\-----
	\begin{verbatim}
	svnadmin load <path> < file.dump
	\end{verbatim}
	Ripristino del repo in <path> dal file.dump
	
	
\end{document}
